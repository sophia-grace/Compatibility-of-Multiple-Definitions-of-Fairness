% Template last modified by Jake Hart; please contact course staff if you have any questions regarding using this template

\documentclass{cisXXX} % You must have the cisXXX .cls file in your project or working directory (i.e. the same directory as this document) 
\usepackage{comment}

\HWauthor{Sophia Trump}{strump@brynmawr.edu} % Put your name and Penn email on this line
\HWno{2} % Enter the number of the homework you are working on
\HWcourse{CIS 399} % Enter the course department and number here
\HWpartner{Paul Brown} % If your class allows group work, put your partners here
\HWpartner{Amelia Earhart} % Otherwise, delete or comment these lines 
\usepackage{amsmath}

\begin{document}
\maketitle
\HWproblem
Show that $(1-FNR) = \frac{A}{A+C}$.

From the definition of equality of false negative rates, we know that $(1-FNR)$ can be rewritten as 
$$1-\frac{C}{A+C}$$

Simplifying the above,
\begin{align*}
1-\frac{C}{A+C} &= \frac{1}{1} - \frac{C}{A+C}\\
&=\frac{1(A+C)}{1(A+C)} - \frac{C}{A+C}\\
&=\frac{A+C}{A+C} - \frac{C}{A+C}\\
&=\frac{A+C-C}{A+C}\\
&=\frac{A}{A+C}
\end{align*}

Thus, from the above, $(1-FNR) = \frac{A}{A+C}$. QED.

\HWproblem
Show that $(1-PPV) = \frac{B}{A+B}$.

From the definition of equality of positive predictive value, we know that $(1-PPV)$ can be rewritten as
$$1-\frac{A}{A+B}$$

Simplifying the above,
\begin{align*}
1-\frac{A}{A+B} &= \frac{1}{1} - \frac{A}{A+B}\\
&= \frac{1(A+B)}{1(A+B} - \frac{A}{A+B}\\
&= \frac{A+B}{A+B} - \frac{A}{A+B}\\
&= \frac{A+B - A}{A+B}\\
&= \frac{B}{A+B}
\end{align*}

Thus, from the above, $(1-PPV) = \frac{B}{A+B}$. QED.

\HWproblem
Show that $\frac{BR}{1-BR} = \frac{A+C}{B+D}$.

From the definition of the base rate of a population, BR can be rewritten as
$$\frac{A+C}{|P|}$$

Substituting the above, $\frac{BR}{1-BR} = \frac{A+C}{B+D}$ becomes
$$\frac{\frac{A+C}{|P|}}{1-\frac{A+C}{|P|}}$$

From the definition of $|P|$, the above can be further rewritten as
$$\frac{\frac{A+C}{A+B+C+D}}{1-\frac{A+C}{A+B+C+D}}$$

Simplifying the above,
\begin{align*}
\frac{\frac{A+C}{A+B+C+D}}{1-\frac{A+C}{A+B+C+D}} &= \frac{\frac{(A+C)(A+B+C+D)}{A+B+C+D}}{1-\frac{(A+C)(A+B+C+D)}{A+B+C+D}}\\
&= \frac{A+C}{(A+B+C+D) - (A+C)}\\
&= \frac{A+C}{B+D}
\end{align*}

Thus, from the above, $\frac{BR}{1-BR} = \frac{A+C}{B+D}$. QED.


\HWproblem
Using the results from Problems 1, 2, and 3, show that:
$$FPR = \frac{BR}{1-BR}\times\frac{1-PPV}{PPV}\times(1 -FNR)$$
.

\HWproblem
The statement from Problem 4 holds for the entire population as well as for each group individually. Suppose that all three fairness notions are satisfied by our hypothesis, i.e. $FPR_1 = FPR_2$, $FNR_1 = FNR_2$, and $PPV_1 = PPV_2$. Further, assume that all of these values, as well as the base rates, are neither 0 nor 1. Show that this implies that the base rates of the groups must be equal.

\HWproblem
Show that if our hypothesis makes no mistakes (i.e. $B_1 = B_2 = 0$ and $C_1 = C_2 = 0$), all three fairness notions will be satisfied, regardless of the base rates for each group.

\end{document}
