\documentclass[11pt]{article}
\usepackage{amsmath,amsfonts,latexsym,graphicx}
\usepackage{xspace}
\usepackage{fullpage,color}
%\usepackage{algo}
\usepackage{url,hyperref}

%\pagestyle{empty}

\setlength{\oddsidemargin}{0in}
\setlength{\topmargin}{0in}
\setlength{\textwidth}{6.5in}
\setlength{\textheight}{8.8in}

\newtheorem{fact}{Fact}
\newtheorem{lemma}{Lemma}
\newtheorem{theorem}[lemma]{Theorem}
\newtheorem{assumption}[lemma]{Assumption}
\newtheorem{corollary}[lemma]{Corollary}
\newtheorem{prop}[lemma]{Proposition}
\newtheorem{claim}[lemma]{Claim}
\newtheorem{remark}[lemma]{Remark}
\newtheorem{prob}{Problem}
\newtheorem{conjecture}{Conjecture}

\begin{document}
\newenvironment{note}[1]{\medskip\noindent \textbf{#1:}}%

        {\medskip}


\newenvironment{proof}{ \vspace{0.05in}\noindent{\bf Proof:}}%
        {\hspace*{\fill}$\Box$\par}
\newenvironment{proofsketch}{\noindent{\bf Proof Sketch.}}%
        {\hspace*{\fill}$\Box$\par\vspace{4mm}}
\newenvironment{proofof}[1]{\smallskip\noindent{\bf Proof of #1.}}%
        {\hspace*{\fill}$\Box$\par}

\newcommand{\etal}{{\em et al.}\ }
\newcommand{\assign}{\leftarrow}
\newcommand{\eps}{\epsilon}
\newcommand{\NP}{\textbf{NP}}

\newcommand{\opt}{\textrm{\sc OPT}}
\newcommand{\script}[1]{\mathcal{#1}}
\newcommand{\ceil}[1]{\lceil #1 \rceil}
\newcommand{\floor}[1]{\lfloor #1 \rfloor}

\newcommand{\expect}{\mbox{\bf E}}
\newcommand{\Var}{\mbox{\bf Var}}

\newcommand{\polylog}{\text{polylog}}


\setlength{\fboxrule}{.5mm}\setlength{\fboxsep}{1.2mm}
\newlength{\boxlength}\setlength{\boxlength}{\textwidth}
\addtolength{\boxlength}{-4mm}
\begin{center}\framebox{\parbox{\boxlength}{\bf
CIS399: The Science of Data Ethics \hfill Homework 2: Feb 26, 2019\\ 
Instructor: Michael Kearns \& Ani Nenkova \hfill Name: Sophia Trump}}\end{center}
\vspace{5mm}

\section{Problem 1}
Show that $(1-FNR) = \frac{A}{A+C}$.

\section{Problem 2}
Show that $(1-PPV) = \frac{B}{A+B}$.

\section{Problem 3}
Show that $\frac{BR}{1-BR} = \frac{A+C}{B+D}$.

\section{Problem 4}
Using the results from Problems 1, 2, and 3, show that:
\[FPR = \frac{BR}{1-BR}\times\frac{1-PPV}{PPV}\times(1 -FNR)\]
.

\section{Problem 5}
The statement from Problem 4 holds for the entire population as well as for each group individually. Suppose that all three fairness notions are satisfied by our hypothesis, i.e. $FPR_1 = FPR_2$, $FNR_1 = FNR_2$, and $PPV_1 = PPV_2$. Further, assume that all of these values, as well as the base rates, are neither 0 nor 1. Show that this implies that the base rates of the groups must be equal.

\section{Problem 6}
Show that if our hypothesis makes no mistakes (i.e. $B_1 = B_2 = 0$ and $C_1 = C_2 = 0$), all three fairness notions will be satisfied, regardless of the base rates for each group.

\subsection{Subsection 2.1} 
This is an example of a subsection.



\end{document}



